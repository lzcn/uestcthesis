% !TeX root = ./main.tex

\uestcsetup{
  title              = {5G移动通信基站天线关键技术及其特征模分析方法研究},
  title*             = {Key Technologies and Characteristic Mode Analysis Methods for 5G Base Station Antennas},
  author             = {张某},
  author*            = {Zhang Mou},
  speciality         = {电磁场与微波技术},
  speciality*        = {Electromagnetic Field and Microwave Technology},
  supervisor         = {李某某, 教授},
  supervisor*        = {Prof. XXX, Prof. XXX},
  % date               = {2017-05-01},  % 默认为今日
  % professional-type  = {专业学位类型},
  % professional-type* = {Professional degree type},
  department         = {电子科技与工程学院},
  student-id         = {2017XXXXXXXX},
  % secret-level       = {秘密},     % 绝密|机密|秘密|控阅,注释本行则公开
  % secret-level*      = {Secret},  % Top secret | Highly secret | Secret
  % secret-year        = {10},      % 保密/控阅期限
  % reviewer           = true,      % 声明页显示“评审专家签名”
  % udc                = {621.39}   % 国际十进分类法 UDC 的类号
  % book-class         = {TN828.6}, % 中图法分类号 http://pss.uestc.edu.cn:8080/chineseSearch.action
  % 数学字体
  math-style         = GB,  % 可选:GB, TeX, ISO
  math-font          = lm,  % 可选:stix, xits, libertinus
}


% 加载宏包

% 定理类环境宏包
\usepackage{amsthm}

% 插图
\usepackage{graphicx}

% 跨页表格
\usepackage{longtable}
\usepackage{tabularx}
\newcolumntype{L}[1]{>{\raggedright\arraybackslash}p{#1}}
\newcolumntype{C}[1]{>{\centering\arraybackslash}p{#1}}
\newcolumntype{R}[1]{>{\raggedleft\arraybackslash}p{#1}}

% 算法
\usepackage[ruled,linesnumbered]{algorithm2e}

% SI 量和单位
\usepackage{siunitx}

% 参考文献使用 BibTeX + natbib 宏包
% 顺序编码制
\usepackage[sort]{natbib}
\bibliographystyle{uestcthesis-numerical}

% 著者-出版年制
% \usepackage{natbib}
% \bibliographystyle{uestcthesis-authoryear}

% 本科生参考文献的著录格式
% \usepackage[sort]{natbib}
% \bibliographystyle{uestcthesis-bachelor}

% 参考文献使用 BibLaTeX 宏包
% \usepackage[style=uestcthesis-numeric]{biblatex}
% \usepackage[bibstyle=uestcthesis-numeric,citestyle=uestcthesis-inline]{biblatex}
% \usepackage[style=uestcthesis-authoryear]{biblatex}
% \usepackage[style=uestcthesis-bachelor]{biblatex}
% 声明 BibLaTeX 的数据库
% \addbibresource{bib/uestc.bib}

% 配置图片的默认目录
\graphicspath{{figures/}}

% 数学命令
\makeatletter
\newcommand\dif{%  % 微分符号
  \mathop{}\!%
  \ifuestc@math@style@TeX
    d%
  \else
    \mathrm{d}%
  \fi
}
\makeatother
\newcommand\eu{{\symup{e}}}
\newcommand\iu{{\symup{i}}}

% 用于写文档的命令
\DeclareRobustCommand\cs[1]{\texttt{\char`\\#1}}
\DeclareRobustCommand\env{\texttt}
\DeclareRobustCommand\pkg{\textsf}
\DeclareRobustCommand\file{\nolinkurl}

% hyperref 宏包在最后调用
\usepackage{hyperref}
