\chapter{格式规范}

\section{语言和表述}
除来华留学研究生、外语学科专业研究生外,\textbf{学位论文用中文撰写},采用国家正式公布实施的简化汉字。经导师同意的,也可用英文撰写,但答辩后须\textbf{另提交与英文版学位论文内容一致的中文版学位论文进行重复率检查},通过后方可提交院学位评定分委员会和校学位评定委员会审议,且\textbf{存档须使用中文版学位论文}。


论文采用的术语、符号、代号,须全文统一,并符合规范化的要求。对于论文中出现的非通用性的新名词、新术语、新概念,应作相应解释。对于反复出现的较长词组,在其\textbf{首次出现时使用中文全称},并在括号内注明英文全称及缩写,例如“电子科技大学(University of Electronic Science and Technology of China, UESTC)”;在此之后,统一使用缩写词代替。

学位论文表述要严谨简明,重点突出,专业常识应简写或不写,做到立论正确、层次分明、数据可靠、文字凝练、说理透彻、推理严谨,避免使用文学性质的带感情色彩的非学术性词语。\textbf{学位论文作者具有唯一性,避免“我们”等用词}。

\section{标题和层次}
论文各章节标题要突出重点、简明扼要,\textbf{不要超过一行},标题中不加标点符号。标题中尽量不采用英文缩写词,必须采用时应使用本行业的通用缩写词。

论文章节层次要清楚,\textbf{一般到三级层级(例如“1.1.1”)即可},最多到四级层次。各章节层次均应有标题,标题由序号和名称组成,之间空1个半角字符。一级标题(章标题)居中书写,章序用中文数字;次级标题顶格书写,节序用阿拉伯数字,阐述内容另起一段书写。

\section{字体和段落}
\textbf{若无特殊说明,论文中的中文统一用宋体,数字和英文用统一用Times New Roman字体。从中文摘要开始,所有文字段落和标题行间距均取固定值20磅;所有段落按两端对齐、首行缩进2个全角字符方式书写内容。}

\textbf{中、英文混排时},除小数点以及引用的分图序号、公式序号等外,宜使用全角标点符号(逗号、冒号、括号、引号等);英文段落中,符号使用应遵循英文书写习惯,统一使用半角符号,并规范使用空格;中文用黑体或加粗的地方,对应数字和英文宜使用加粗Times New Roman字体。

中、英文字号对应关系如表2-1所示,主要文字及段落格式要求如表2-2所示。

\begin{table}[!h]
  \centering
  \caption{中、英文字号对应关系}
  \label{tab:fontsize}
  \begin{tabular*}{\linewidth}{C{0.22\linewidth}C{0.22\linewidth}C{0.22\linewidth}C{0.22\linewidth}}
    \toprule
    中文字号 & 英文磅数 & 中文字号 & 英文磅数  \\
    \midrule
    二号     & 22     & 四号 & 14 \\
    小二     & 18     & 小四 & 12 \\
    三号     & 16     & 五号 & 10.5 \\
    小三     & 15     & 小五 & 9 \\
    \bottomrule
  \end{tabular*}
\end{table}

\begin{table}[!h]
  \centering
  \caption{中、英文字号对应关系}
  \label{tab:format}
  \begin{tabular*}{\linewidth}{
    @{\extracolsep{\fill}}C{0.12\linewidth} % 内容
    @{\extracolsep{\fill}}C{0.05\linewidth} % 字体
    @{\extracolsep{\fill}}C{0.06\linewidth} % 字号
    @{\extracolsep{\fill}}C{0.14\linewidth} % 对齐方式
    @{\extracolsep{\fill}}C{0.08\linewidth} % 段前距
    @{\extracolsep{\fill}}C{0.08\linewidth} % 段后距
    @{\extracolsep{\fill}}C{0.42\linewidth} % 示例或备注
  }
  \toprule
  内容 & 字体 & 字号 & 对齐方式 & 段前距 & 段后距 & 示例或备注 \\
  \midrule
  一级标题 & 黑体 & 小三 & 居中 & 24磅 & 18磅 & 第一章 绪论 \\
  二级标题 & 黑体 & 四号 & 顶格左对齐 & 18磅 & 6磅 & 3.2 实验装置和方法 \\
  三级标题 & 黑体 & 四号 & 顶格左对齐 & 12磅 & 6磅 & 4.1.2 测试结果 \\
  四级标题 & 黑体 & 小四 & 顶格左对齐 & 12磅 & 6磅 & 5.3.4.1 协商系统 \\
  正文 & \textsuperscript{*} & 小四 & 两端对齐(首行缩进)& 0磅 & 0磅 & {\textsuperscript{*}\kaishu\fontsize{9bp}{9bp}\selectfont 未注明字体的,统一按“中文宋体,英文、数字Times New Roman”原则} \\
  页眉 & & 五号 & 居中 & 0磅 & 0磅 & \\
  页码 & & 小五 & 居中 & 0磅 & 0磅 &  \\
  脚注 & & 小五 & 两端对齐 & 0磅 & 0磅 & \\
  参考文献 & & 五号 & 两端对齐(悬挂缩进)& 0磅 & 0磅 & \\
  附录 & & 五号 & \textsuperscript{*} & 0磅 & 0磅 & {\textsuperscript{*}\kaishu\fontsize{9bp}{9bp}\selectfont 根据附录形式选择合适的排版方式。} \\
  图片 & & 五号\textsuperscript{*} & 居中 & 6磅 & 0磅 & {\textsuperscript{*}\kaishu\fontsize{9bp}{9bp}\selectfont 图中文字显示大小跟图题文字一致。} \\
  图题 & & 五号 & 居中\textsuperscript{*} & 6磅 & 12磅 & {\textsuperscript{*}\kaishu\fontsize{9bp}{9bp}\selectfont 超过一行的图题并非居中,详见2.4.1} \\
  表格 & & 五号 & 居中 & 0磅 & 6磅 & {\kaishu\fontsize{9bp}{9bp}\selectfont 一般采用三线表样式} \\
  表题 & & 五号 & 居中\textsuperscript{*} & 12磅 & 6磅 & {\textsuperscript{*}\kaishu\fontsize{9bp}{9bp}\selectfont 超过一行的表题并非居中,详见2.4.2} \\
  图表附注 & & 五号 & 顶格& 6磅 &6磅  \\
  公式 & & 小四 & 居中 &6磅 &6磅  \\
  公式编号 & & 小四 & 右对齐\textsuperscript{*} & 6磅 & 6磅 & {\textsuperscript{*}\kaishu\fontsize{9bp}{9bp}\selectfont 公式编号前不加引导线,详见2.5} \\
\bottomrule
\end{tabular*}
\vspace{6bp}
\end{table}

其他要求:
\begin{enumerate}
    \item 各级标题不得置于页面的最后一行,即须与下段同页;
    \item 两个标题之间无正文时,第二个标题的段前距设置为0磅;
    \item 图、表、公式统一采用单倍行距;
    \item 只有一、两行文字的,不得单独作为一页内容;
    \item 除各章最后一页外,中间页面不得出现较大空白;
    \item 必要时,可在规定的格式要求基础上适当微调,以利于排版,但显示效果不得与规定的格式要求存在明显差距。
\end{enumerate}

表应有自明性。每个表应有简短确切的表题,五号字,居中置于表的正上方。表题超过一行时,两端对齐,左右缩进4字符(类似图2-1标题)。

表题的段前距12磅,段后距6磅;表格之后首段正文的段前距6磅。若有附注,用五号字顶格写在表下方,首段段前距、末段段后距设为6磅。

表格采用三线表样式,上下边线线宽1.5磅,表内线条线宽0.75磅,必要时可加辅助线。表内文字五号字、单倍行距、上下居中,行高0.6 cm左右为宜。

表中数据应准确填写,不得使用“同上”、“同左”等表述。表中“空白”代表未测或无此项,“…”代表未发现,“0”代表结果确为零。

表格一般不跨页编排,仅当一页内编排不下时才可转页,以续表形式接排,续表应重复表头和关于单位的陈述,并在表题结尾以“(续)”注明,例如:表3-2 加入激素后的实验结果比较(续)。

